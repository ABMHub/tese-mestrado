Garantir conformidade documental requer uma identificação acurada dos documentos, que também serve o propósito de manter o dado consistente ao decorrer das esteiras de verificação. 
Grande parte dos estudos lidam com a identificação como uma tarefa de classificação documental, ou como uma tarefa de segmentação. Entretanto, documentos industriais estão sempre mudando sua forma, e os modelos que os classificam precisam de constantes atualizações. Nesses casos, avaliar se determinado documento está de acordo com o histórico de documentos aceitos é uma abordagem mais apropriada.
Esta tese adentra no problema de comparar a aparência de dois (ou mais) documentos para determinar se eles dividem ou não a mesma disposição de informações. Portanto, esse problema é atacado com o paradigma \acrit{ZSL}, que é uma abordagem poderosa para cenários onde as classes encontradas na inferência não coincidem com as classes usadas no treino. Para dar suporte ao estudo, o \acrit{LA-CDIP} é introduzido, um dataset contendo 4,993 documentos, distribuídos por 144 classes, reorganizadas a partir da base de dados \acrit{RVL-CDIP}, realizando uma separação prioritariamente sintática, ao invés de semântica. Essa abordagem é testada usando redes siamesas e \textit{Contrastive Learning} através de muitas arquiteturas neurais conhecidas, incluindo ResNet, EfficientNet e \acrit{ViT}. Em cenários \acrshort{ZSL}, o método proposto atinge um \acrit{EER} abaixo de 5\% na verificação com validação cruzada. Além disso, a abordagem \acrit{VDM} performa com maior precisão que \acritpl{LLM} de código aberto e rivaliza contra o modelo GPT-4o, da OpenAI, demonstrando a superioridade de uma técnica especialista sobre modelos multimodais generalistas. Essas descobertas mostram que a abordagem proposta mantém alta acurácia enquanto usa significativamente menos parâmetros que \acrshort{LLM}s, demonstrando um uso mais prático para aplicações de conformidade documental na indústria.
