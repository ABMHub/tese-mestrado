Ensuring document compliance requires accurate document identification, which plays a crucial role in maintaining consistency throughout document analysis pipelines. Several studies approach layout identification as a document image classification or segmentation task. However, due to the ever-changing nature of industry documents, a traditional classification with entropy learning is often insufficient, as models require frequent retraining. In these cases, determining whether two or more documents share the same visual layout is a more suitable approach.
In this paper, we address the problem of matching the visual appearance of two (or more) documents to determine whether they share the same layout. To achieve this, we adopt a \acrfull{ZSL} approach, a powerful technique for scenarios where training classes do not align with those encountered during inference. %Despite its potential, \gls{ZSL} remains underexplored in document layout understanding.
To support our study, we introduce \acrfull{LA-CDIP}, a dataset comprising 4,993 documents across 144 classes, which we reorganized from the \acrfull{RVL-CDIP} database to emphasize visual structure over semantic content. We benchmark our approach using a siamese network and contrastive learning framework across multiple backbone architectures, including ResNet, EfficientNet, and \acrfull{ViT}.
In zero-shot scenarios, our method achieves an \acrfull{EER} below 5\% in 1-vs-1 verification with cross-validation. Furthermore, our \acrfull{VDM} approach outperforms lighter \acrfullpl{LLM} and rivals GPT-4o, highlighting the superiority of specialized techniques over general-purpose multimodal models. These findings show that our approach maintains high accuracy while using significantly fewer parameters than large multimodal models, making it more practical for real-world document compliance applications.

% The main goal of this work is to study the use of neural networks on the problem of lipreading complete sentences. This work is based on LipNet and LCANet, two neural network architectures based on tridimensional convolutions and recurrent networks. This study is based on a series of ablations over the pre-processing, architecture choices and the use of a language model while decoding the model's output. THis work shows that using a dynamic crop, opposing to a fixed area crop, yields worse results, between 14\% and 16\% increase in \acrshort{WER}. LCANet outperformed Lipnet, achieving up to 36\% better accuracy and a faster convergence, obtaining 4 points of loss in the 23th epoch, 29 epochs faster than LipNet. Finally, the language model brought a strictly better accuracy over every test case.

% O \emph{abstract} é o resumo feito na língua Inglesa. Embora o conteúdo apresentado
% deva ser o mesmo, este texto não deve ser a tradução literal de cada palavra ou
% frase do resumo, muito menos feito em um tradutor automático. É uma língua
% diferente e o texto deveria ser escrito de acordo com suas nuances (aproveite para ler
% \url{http://dx.doi.org/10.6061%2Fclinics%2F2014(03)01}). Por exemplo: \emph{This work presents useful information on how to create a scientific text to describe
% and provide examples of how to use the Computer Science Department's \LaTeX\ class. The \unbcic\
% class defines a standard format for texts, simplifying the process of generating
% CIC documents and enabling authors to focus only on content. The standard was approved
% by the Department's professors and used to create this document. Future work includes
% continued support for the class and improvements on the explanatory text.}